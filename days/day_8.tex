\begin{flushright}
    \section*{\Large{Day 8 八日目 \\
    Kata Kerja Bentuk Nai dan Kegunaannya}}
    \addcontentsline{toc}{section}{
        Day 8 \begin{CJK}{UTF8}{ipxm}八日目\end{CJK}: 
        Kata Kerja Bentuk Nai dan Kegunaannya
    }
\end{flushright}

\subsection*{
    ~ないでください \\ 
    Larangan untuk tidak melalukan sesuatu
}
\addcontentsline{toc}{subsection}{
    \begin{CJK}{UTF8}{ipxm}~ないでください\end{CJK}: 
    Larangan untuk tidak melalukan sesuatu
}
Larangan yang berbentuk perintah.\\
Bentuk:
\begin{enumerate}
    \item Kata Kerja Bentuk Nai (ない) + で + ください
\end{enumerate}
Contoh: 
\begin{enumerate}
    \item そのプールで泳がないでください。
    \\ Tolong jangan berenang di kolam itu.
    \item あのホテルに入らないでください。
    \\ Tolong jangan masuk ke hotel itu.
    \item そのケーキをたべないでください。
    \\ Tolong jangan makan kue itu.
\end{enumerate}

\vspace{0.2cm}\hrule height 1pt\vspace{0.2cm}

% =====================================================

\subsection*{
    ~なければなりません \\
    Menjelaskan ungkapan keharusan melakukan sesuatu
}
\addcontentsline{toc}{subsection}{
    \begin{CJK}{UTF8}{ipxm}~なければなりません\end{CJK}: 
    Menjelaskan ungkapan keharusan melakukan sesuatu
}
Bentuk:
\begin{enumerate}
    \item Kata Kerja Bentuk Nai (な\sout{い}) + ければなりません。
\end{enumerate}
Contoh: 
\begin{enumerate}
    \item あさってのしけんを準備しなければなりません。
    \\ (Saya) harus mempersiapkan ujian pada lusa hari.
    \item きょう、早く帰らければなりません。
    \\ (Saya) hari ini harus pulang awal.
    \item あした、早くおきなければなりません。
    \\ (Saya) besok harus bangun awal.
\end{enumerate}

\vspace{0.2cm}\hrule height 1pt\vspace{0.2cm}

% =====================================================
\newpage
\subsection*{
    ~なくてもいいです \\
    Menjelaskan ungkapan kebolehan tidak melakukan sesuatu
}
\addcontentsline{toc}{subsection}{
    \begin{CJK}{UTF8}{ipxm}~なくてもいいです\end{CJK}: 
    Menjelaskan ungkapan kebolehan tidak melakukan sesuatu
}
Bentuk:
\begin{enumerate}
    \item Kata Kerja Bentuk Nai (な\sout{い}) + ければなりません。
\end{enumerate}
Contoh: 
\begin{enumerate}
    \item あさってのパーティをじゅんびしなくてもいいです。
    \\ Tidak menyiapkan pesta untuk lusa pun tidak apa-apa.
    \item シャツを着なくてもいいです。
    \\ Tidak memakai kemeja pun boleh.
    \item ネクタイをしめなくてもいいです。
    \\ Tidak memakai dasi pun tidak apa-apa.
\end{enumerate}

\vspace{0.2cm}\hrule height 1pt\vspace{0.2cm}

% =====================================================

\subsection*{
    ~たら \\
    Menyatakan pengandaian
}
\addcontentsline{toc}{subsection}{
    \begin{CJK}{UTF8}{ipxm}~たら\end{CJK}: 
    Menyatakan pengandaian
}
Dapat diartikan "kalau...terjadi, maka...terjadi".\\
Bentuk:
\begin{enumerate}
    \item Kata Kerja (た形) + ら
    \item Kata Sifat bentuk た + ら
\end{enumerate}
Contoh: 
\begin{enumerate}
    \item 駅に来たら、電話してください。
    \\ Kalau sudah sampai stasiun, tolong telfon (saya).
    \item タクシーで行ったら、払わなければなりません。
    \\ Kalau pergi dengan taksi, harus membayar.
    \item 料理がおいしかったら、食べます。
    \\ Kalau masakannya enak, (saya) akan makan.
\end{enumerate}

\vspace{0.2cm}\hrule height 1pt\vspace{0.2cm}


