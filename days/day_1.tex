\begin{flushright}
    \section*{\Large{Day 1 一日目 \\
    Pola Kalimat Dasar Jepang}}
    \addcontentsline{toc}{section}{Day 1 \begin{CJK}{UTF8}{ipxm}一日目\end{CJK}: Pola Kalimat Dasar Jepang}
\end{flushright}

\subsection*{
    ~は~です \\ 
    Menunjukan topik/subjek kalimat
}
\addcontentsline{toc}{subsection}{
    \begin{CJK}{UTF8}{ipxm}~は~です\end{CJK}: 
    Menunjukan topik/subjek kalimat
}
Partikel "wa" (は) merupakan penanda topik atau subjek terhadap kosakata di depannya, kemudian
kosakata di belakangnya adalah predikat. Dalam bahasa Indonesia bisa diartikan "adalah".\\
Bentuk:
\begin{enumerate}
    \item Kata Benda 1 + は + Kata Benda 2 + です
\end{enumerate}
Contoh: 
\begin{enumerate}
    \item わたしはよしおかかなです
    \\ Saya adalah Yoshioka Kana.
    \item こちらはちちです。
    \\ Ini adalah ayah (saya).
    \item ちちはよしおかマリオです。
    \\ Ayah (saya) adalah Mario Yoshioka.
\end{enumerate}

\vspace{0.2cm}\hrule height 1pt\vspace{0.2cm}

% =====================================================

\subsection*{
    ~は~ではありません/~じゃありません \\
    Menunjukan topik/subjek kalimat negatif
}
\addcontentsline{toc}{subsection}{
    \begin{CJK}{UTF8}{ipxm}~は~ではありません/~じゃありません\end{CJK}: 
    Menunjukan topik/subjek kalimat negatif
}
Pola kalimat ini merupakan bentu negatif dari ~は~です.\\
Bentuk:
\begin{enumerate}
    \item Kata Benda 1 + は + Kata Benda 2 + ではありません/~じゃありません
\end{enumerate}
Contoh: 
\begin{enumerate}
    \item ちちは日本人ではありません。ポンゴランド人です。
    \\ Ayah (saya) bukan orang Jepang. Orang Pongoland.
    \item ちちは50さいじゃありません。55さいです。
    \\ Ayah (saya) tidak berusia 50 tahun, 55 tahun.
\end{enumerate}

\vspace{0.2cm}\hrule height 1pt\vspace{0.2cm}

% =====================================================
\newpage
\subsection*{
    ~は~ですか \\
    Menunjukan topik/subjek kalimat tanya
}
\addcontentsline{toc}{subsection}{
    \begin{CJK}{UTF8}{ipxm}~は~ですか\end{CJK}: 
    Menunjukan topik/subjek kalimat tanya
}
Pola kalimat berikut ini adalah pola kalimat dasar bahasa Jepang.
Dimana partikel "wa" (は) merupakan penanda topik atau subjek terhadap 
kosakata di depannya, kemudian kosakata dibelakangnya adalah predikat 
dari kalimat tersebut namun berbentuk tanya.\\
Bentuk:
\begin{enumerate}
    \item Kata Benda 1 + は + Kata Benda 2 + ですか
\end{enumerate}
Contoh: 
\begin{enumerate}
    \item あなたはだれですか?
    \\ Anda siapa?
    \item あなたわ日本人ですか?
    \\ Apakah anda orang Jepang?
\end{enumerate}

\vspace{0.2cm}\hrule height 1pt\vspace{0.2cm}

% =====================================================

\subsection*{
    ~も~ \\
    Partikel yang berarti "juga"
}
\addcontentsline{toc}{subsection}{
    \begin{CJK}{UTF8}{ipxm}~も~\end{CJK}: 
    Partikel yang berarti "juga"
}
Partikel "mo" (も) memiliki arti "juga", dimana 2 subjek memiliki 
keterangan/predikat yang sama.\\
Bentuk:
\begin{enumerate}
    \item Kata Benda 1 + も + Kata Benda 2 + です
\end{enumerate}
Contoh: 
\begin{enumerate}
    \item おはようございます。ポンゴさん、お父さんも先生ですか?
    \\ Selamat pagi. Pongo, apakah ayahmu juga guru?
    \item いもうとも会社員です。
    \\ Adik perempuan (saya) juga pegawai perusahaan.
    \item わたしは学生です。いもうとも学生です。
    \\ Saya adalah pelajar. Adik perempuan saya juga pelajar.
\end{enumerate}

\vspace{0.2cm}\hrule height 1pt\vspace{0.2cm}

% =====================================================
\newpage
\subsection*{
    ~さい \\
    Menyatakan umur/usia
}
\addcontentsline{toc}{subsection}{
    \begin{CJK}{UTF8}{ipxm}~さい\end{CJK}: 
    Menyatakan umur/usia
}
Umur dalam bahasa jepang disebut toshi (年) atau nenrei (年齢).
Adapun penyebutan umur atau usia dalam bahasa Jepang adalah dengan 
menyebutkan angka/bilangan terlebih dahulu, kemudian ditambahkan sai, 
yaitu satuan yang digunakan untuk umur/usia.\\
Bentuk:
\begin{enumerate}
    \item Kata Benda 1 + は + (bilangan/angka) + さい + です
\end{enumerate}
Contoh: 
\begin{enumerate}
    \item わたしは22さいです。
    \\ Saya 22 tahun.
    \item 母は50さいです。
    \\ Ibu (saya) 50 tahun.
\end{enumerate}

\vspace{0.2cm}\hrule height 1pt\vspace{0.2cm}

% =====================================================

\subsection*{
    ~さいですか/おいくつですか \\
    Menanyakan umur/usia
}
\addcontentsline{toc}{subsection}{
    \begin{CJK}{UTF8}{ipxm}~さいですか/おいくつですか\end{CJK}: 
    Menanyakan umur/usia
}
Nansai desuka (なんさいですか) dan oikutsu desuka (おいくつですか) 
mempunyai arti yang sama. Namun, oikutsu desika lebih sopan dari nansai desuka.\\
Bentuk:
\begin{enumerate}
    \item Kata Benda 1 + は + なんさい + ですか
    \item Kata Benda 1 + は + おいくつ + ですか
\end{enumerate}
Contoh: 
\begin{enumerate}
    \item あなたはなんさいですか?
    \\ Berapa umur anda?
    \item マリオさんはおいくつですか?
    \\ Mario usianya berapa?
\end{enumerate}

\vspace{0.2cm}\hrule height 1pt\vspace{0.2cm}

% =====================================================
\newpage
\subsection*{
    これ/それ/あれ \\
    Menunjuk benda yang tidak diketahui namanya
}
\addcontentsline{toc}{subsection}{
    \begin{CJK}{UTF8}{ipxm}これ/それ/あれ\end{CJK}: 
    Menunjuk benda yang tidak diketahui namanya
}
Pola kalimat ini menyatakan kata tunjuk "ini", "itu", "itu" (jauh).\\
Bentuk:
\begin{enumerate}
    \item これ/それ/あれ + は + Kata Benda + さい + です
\end{enumerate}
Contoh: 
\begin{enumerate}
    \item これはじしょです。
    \\ Ini adalah kamus.
    \item それは本です。
    \\ Itu adalah buku
    \item あれはきょうかしょです。
    \\ Itu (jauh) adalah buku pelajaran.
\end{enumerate}

\vspace{0.2cm}\hrule height 1pt\vspace{0.2cm}

% =====================================================

\subsection*{
    この/その/あの \\
    Menunjuk benda yang spesifik
}
\addcontentsline{toc}{subsection}{
    \begin{CJK}{UTF8}{ipxm}この/その/あの\end{CJK}: 
    Menunjuk benda yang spesifik
}
Pola kalimat ini menyatakan kata tunjuk yang menyertakan kata 
bendanya, yang berarti "benda ini/itu".\\
Bentuk:
\begin{enumerate}
    \item この/その/あの + Kata Benda
\end{enumerate}
Contoh: 
\begin{enumerate}
    \item このいすはだれのですか?
    \\ Kursi ini milik siapa?
    \item そのつくえもおじいさんのでしか?
    \\ Meja itu juga miliki kakek(mu)?
    \item あのベッドはわたしのです。
    \\ Tempat tidur ini adalah milik saya.
\end{enumerate}

\vspace{0.2cm}\hrule height 1pt\vspace{0.2cm}

% =====================================================
\newpage
\subsection*{
    ここ/そこ/あこ \\
    Kata tunjuk tempat
}
\addcontentsline{toc}{subsection}{
    \begin{CJK}{UTF8}{ipxm}ここ/そこ/あこ\end{CJK}: 
    Kata tunjuk tempat
}
Pola kalimat ini menyatakan kata tunjuk tempat "sini", "situ", "sana".\\
Bentuk:
\begin{enumerate}
    \item ここ/そこ/あそこ + は + Kata Benda (nama tempat) + です
\end{enumerate}
Contoh: 
\begin{enumerate}
    \item ここはにわです。
    \\ Disini adalah halaman.
    \item そこはガレージです。
    \\ Disitu adalah garasi.
    \item といれはあそこです。
    \\ Toilet disana.
\end{enumerate}

\vspace{0.2cm}\hrule height 1pt\vspace{0.2cm}