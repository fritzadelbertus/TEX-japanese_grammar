\begin{flushright}
    \section*{\Large{Day 9 九日目 \\
    Pola Urutan Waktu dan Bentuk Lampau}}
    \addcontentsline{toc}{section}{
        Day 9 \begin{CJK}{UTF8}{ipxm}九日目\end{CJK}: 
        Pola Urutan Waktu dan Bentuk Lampau
    }
\end{flushright}

\subsection*{
    ~時 \\ 
    Menjelaskan waktu terjadinya aktifitas atau kejadian
}
\addcontentsline{toc}{subsection}{
    \begin{CJK}{UTF8}{ipxm}~時\end{CJK}: 
    Menjelaskan waktu terjadinya aktifitas atau kejadian
}
Bentuk:
\begin{enumerate}
    \item Kata Kerja Bentuk Biasa (ふつう形) + 時
\end{enumerate}
Contoh: 
\begin{enumerate}
    \item 会うときレポートも出します。
    \\ Saat bertemu akan saya serahkan laporannya.
    \item 訪問した時にレポートをだしました。
    \\ Saat berkungjung (kemarin) saya sudah serahkan laporannya.
    \item 出すときサインをもらいました。
    \\ Saat akan menyerahkan saya mendapat tanda tangan.
\end{enumerate}

\vspace{0.2cm}\hrule height 1pt\vspace{0.2cm}

% =====================================================

\subsection*{
    ~まえに \\
    Menjelaskan kejadian beberapa waktu yang lalu
}
\addcontentsline{toc}{subsection}{
    \begin{CJK}{UTF8}{ipxm}~まえに\end{CJK}: 
    Menjelaskan kejadian beberapa waktu yang lalu
}
Bentuk:
\begin{enumerate}
    \item Kata Kerja Bentuk Kamus (辞書形) + まえに。
    \item Kata Benda + の + まえに。
\end{enumerate}
Contoh: 
\begin{enumerate}
    \item リーさんは仕事の前に何をしますか?
    \\ Lee melakukan apa sebelum (mulai) bekerja?
    \item 仕事する前に机を掃除しますよ。
    \\ Sebelum mulai bekerja, membersihkan meja.
    \item 家を出る前に時々新聞をよみます。
    \\ Sebelum keluar rumah kadang-kadang baca koran.
\end{enumerate}

\vspace{0.2cm}\hrule height 1pt\vspace{0.2cm}

% =====================================================
\newpage
\subsection*{
    ~までに \\
    Menerangkan Batas Akhir
}
\addcontentsline{toc}{subsection}{
    \begin{CJK}{UTF8}{ipxm}~までに\end{CJK}: 
    Menerangkan Batas Akhir
}
Bentuk:
\begin{enumerate}
    \item Waktu + までに
\end{enumerate}
Contoh: 
\begin{enumerate}
    \item 大会は来週の水曜日ですか、日曜日までに覚えなければなりません。
    \\ Karena kontesnya Rabu pekan depan, harus hafal paling lambat hari minggu.
    \item あ、忘れました!スピーチテキストは明日までに出さなければなりません。
    \\ Ah, lupa! Teks pidatonya harus dikumpulkan paling lambat besok.
\end{enumerate}

\vspace{0.2cm}\hrule height 1pt\vspace{0.2cm}

% =====================================================

\subsection*{
    ~てから \\
    Menerangkan tindakan yang dilakukan setelah melakukan sesuatu
}
\addcontentsline{toc}{subsection}{
    \begin{CJK}{UTF8}{ipxm}~てから\end{CJK}: 
    Menerangkan tindakan yang dilakukan setelah melakukan sesuatu
}
Bentuk:
\begin{enumerate}
    \item Kata Kerja 1 (て形) + から + Kata Kerja 2
\end{enumerate}
Contoh: 
\begin{enumerate}
    \item リーさんは手を洗ってから食べますか?
    \\ Lee tangannya dicuci dulu baru makan ya?
    \item もちろん!汚いから手を洗ってから食べるよ!
    \\ Iyalah! Karena kotor tangan dicuci baru makan ya!
    \item いいえ、一時間ぐらい待ってから歯を磨きます。
    \\ Tidak, setelah menunggu sekitar 1 jam saya menyikat gigi.
    \item 私はズボンをぬいでからあびます。
    \\ Saya lepas celana dulu setelah itu mandi.
\end{enumerate}

\vspace{0.2cm}\hrule height 1pt\vspace{0.2cm}

% =====================================================
\newpage
\subsection*{
    ~たことがあります \\
    Menjelaskan hal yang pernah dilakukan/dialami
}
\addcontentsline{toc}{subsection}{
    \begin{CJK}{UTF8}{ipxm}~たことがあります\end{CJK}: 
    Menjelaskan hal yang pernah dilakukan/dialami
}
Bentuk:
\begin{enumerate}
    \item Kata Kerja Bentuk Lampau (た形) + こと + が + あります
\end{enumerate}
Contoh: 
\begin{enumerate}
    \item 日本のお酒を飲んだんことがあります。
    \\ Apakah kamu pernah minum sake jepang?
    \item お茶を飲んだこともありますか?
    \\ Apakah pernah minum teh hijau juga?
    \item お寿司も食べたことがあります。イカとサーモンが好きです。
    \\ Iya, aku juga pernah maka sushi. Suka salmon dan cumi.
\end{enumerate}

\vspace{0.2cm}\hrule height 1pt\vspace{0.2cm}
