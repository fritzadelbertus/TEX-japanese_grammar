\begin{flushright}
    \section*{\Large{Day 7 七日目 \\
    Kata Kerja Bentuk Kamus dan Kegunaannya}}
    \addcontentsline{toc}{section}{
        Day 7 \begin{CJK}{UTF8}{ipxm}七日目\end{CJK}: 
        Kata Kerja Bentuk Kamus dan Kegunaannya
    }
\end{flushright}

\subsection*{
    ~こと \\ 
    Menyatakan hobi
}
\addcontentsline{toc}{subsection}{
    \begin{CJK}{UTF8}{ipxm}~こと\end{CJK}: 
    Menyatakan hobi
}
こと mengubah kata kerja menjadi kata benda. 趣味 (しゅみ) adalah hobi.\\
Bentuk:
\begin{enumerate}
    \item Kata Kerja Bentuk Kamus (辞書形) + こと
\end{enumerate}
Contoh: 
\begin{enumerate}
    \item ポンゴさんの趣味はギターをひくことです。
    \\ Hobi Pongo adalah bermain gitar.
    \item メイリンさんお趣味は歌を歌うことです。
    \\ Hobi Meilin adalah menyanyikan sebuah lagu.
    \item わたしの趣味は泳ぐことです。
    \\ Hobi saya adalah berenang.
\end{enumerate}

\vspace{0.2cm}\hrule height 1pt\vspace{0.2cm}

% =====================================================

\subsection*{
    ~がわかります \\
    Menjelaskan tingkat pemahaman terhadap suatu hal
}
\addcontentsline{toc}{subsection}{
    \begin{CJK}{UTF8}{ipxm}~がわかります\end{CJK}: 
    Menjelaskan tingkat pemahaman terhadap suatu hal
}
Bentuk:
\begin{enumerate}
    \item Kata Benda + が(よく、だいたい) わかります
    \item Kata Benda + が(ぜんぜん) わかりません
\end{enumerate}
Contoh: 
\begin{enumerate}
    \item 英語がよくわかります。
    \\ (Saya) mengerti bahasa Inggris dengan baik.
    \item スペイン語がだいたいわかります。
    \\ (Saya) tidak begitu mengerti bahasa Spanyol
    \item アラビア語がぜんぜんわかりません。
    \\ (Saya) sama sekali tidak mengerti bahasa Arab.
\end{enumerate}

\vspace{0.2cm}\hrule height 1pt\vspace{0.2cm}

% =====================================================
\newpage
\subsection*{
    ~ことができます \\
    Menunjukkan kemampuan atau hal yang bisa dilakukan
}
\addcontentsline{toc}{subsection}{
    \begin{CJK}{UTF8}{ipxm}~て\end{CJK}: 
    Menunjukkan kemampuan atau hal yang bisa dilakukan
}
Bentuk:
\begin{enumerate}
    \item Kata Kerja (辞書形) + ことができます
\end{enumerate}
Contoh: 
\begin{enumerate}
    \item ギターをひくことができます。
    \\ Saya bisa bermain gitar.
    \item 歌を歌うことができます。
    \\ Saya bisa bernyanyi nyanyian (menyanyikan lagu).
    \item 日本語を話すことができます。
    \\ Saya bisa berbicara dalam bahasa Jepang.
\end{enumerate}

\vspace{0.2cm}\hrule height 1pt\vspace{0.2cm}

% =====================================================

\subsection*{
    ~がほしい \\
    Menyatakan keinginan
}
\addcontentsline{toc}{subsection}{
    \begin{CJK}{UTF8}{ipxm}~がほしい\end{CJK}: 
    Menyatakan keinginan
}
Bentuk:
\begin{enumerate}
    \item Kata Benda + が + ほしい
\end{enumerate}
Contoh: 
\begin{enumerate}
    \item わたしはカメラがほしいです。
    \\ Saya menginginkan kamera.
    \item いもうとはペットがほしいです。
    \\ Adik perempuan menginginkan binatang peliharaan.
    \item ヒラさんは EXO のアルバムがほしいです。
    \\ Hira menginginkan album EXO
\end{enumerate}

\vspace{0.2cm}\hrule height 1pt\vspace{0.2cm}

% =====================================================
\newpage
\subsection*{
    ~たい \\
    Menyatakan hal yang ingin dilakukan
}
\addcontentsline{toc}{subsection}{
    \begin{CJK}{UTF8}{ipxm}~たい\end{CJK}: 
    Menyatakan hal yang ingin dilakukan
}
Perubahan ke bentuk lampau dan negatif mengikut perubahan kata sifat (い).\\
Bentuk:
\begin{enumerate}
    \item Kata Kerja (\sout{ます}) + たい
\end{enumerate}
Contoh: 
\begin{enumerate}
    \item ポンゴさんは何を食べたいですか?
    \\ Pongo ingin makan apa?
    \item わたしはこのえいがをみたいです。
    \\ Saya ingin melihat film ini.
    \item あのおかしをかいたいです。
    \\ Ingin membeli kue (makanan manis) itu.
\end{enumerate}

\vspace{0.2cm}\hrule height 1pt\vspace{0.2cm}

