\begin{flushright}
    \section*{\Large{Day 11 十一日目 \\
    Pola Bentuk Hubungan Sebab Akibat}}
    \addcontentsline{toc}{section}{
        Day 11 \begin{CJK}{UTF8}{ipxm}十一日目\end{CJK}: 
        Pola Bentuk Hubungan Sebab Akibat
    }
\end{flushright}

\subsection*{
    ~から \\ 
    Menyatakan alasan
}
\addcontentsline{toc}{subsection}{
    \begin{CJK}{UTF8}{ipxm}~から\end{CJK}: 
    Menyatakan alasan
}
Bentuk:
\begin{enumerate}
    \item Kalimat 1 + から + Kalimat 2
\end{enumerate}
Contoh: 
\begin{enumerate}
    \item あしたは休みですから、こんばんアニメを見ます。
    \\ Karena besok libur, malam ini (aku) akan menonton anime.
    \item 日曜日も休みですから、さんぽします。
    \\ Karena hari Minggu juga libur, aku akan berjalan-jalan.
    \item 買い物がすきですから、ようデパートに行きます。
    \\ Karena suka berbelanja, (aku) sering pergi ke swalayan.
\end{enumerate}

\vspace{0.2cm}\hrule height 1pt\vspace{0.2cm}

% =====================================================

\subsection*{
    ~ほうがいい \\
    Memberi saran kepada lawan bicara
}
\addcontentsline{toc}{subsection}{
    \begin{CJK}{UTF8}{ipxm}~ほうがいい\end{CJK}: 
    Memberi saran kepada lawan bicara
}
Bentuk:
\begin{enumerate}
    \item Kata Kerja Ta/Nai (た形/ない形) + ほうがいい
\end{enumerate}
Contoh: 
\begin{enumerate}
    \item あそこの交番に行ったほうがいいですよ。
    \\ Sebaiknya (kita) pergi ke pos polisi itu.
    \item お父さんは疲れているから、車に乗らないほうがいいですよ。
    \\ Karena ayah sedang lelah, sebaiknya tidak naik mobil.
    \item あしたはしけんがあるから、はやく寝たほうがいいですよ。
    \\ Karena besok ada ujian, sebaiknya tidur awal.
\end{enumerate}

\vspace{0.2cm}\hrule height 1pt\vspace{0.2cm}

% =====================================================
\newpage
\subsection*{
    ~のほうが~より \\
    Memberikan perbandingan antara dua hal
}
\addcontentsline{toc}{subsection}{
    \begin{CJK}{UTF8}{ipxm}~のほうが~より\end{CJK}: 
    Memberikan perbandingan antara dua hal
}
Bentuk:
\begin{enumerate}
    \item Kata Benda 1 + のほうが + Kata Benda 2 + より
\end{enumerate}
Contoh: 
\begin{enumerate}
    \item 窓ぎわのほうが真ん中より外が見ます。
    \\ Di pinggir jendela lebih kelihatan daripada di tengah-tengah.
    \item 低いまくらのほうが高いまくらより体にいい。
    \\ Bantal yang rendah lebih baik untuk tubuh dari pada bantal yang tinggi.
    \item やさいのほうがケーキより体にいいです。
    \\ Sayur lebih baik untuk tubuh dari pada kue.
\end{enumerate}

\vspace{0.2cm}\hrule height 1pt\vspace{0.2cm}

% =====================================================

\subsection*{
    ~つもりです \\
    Menunjukkan kemauan dan keinginan
}
\addcontentsline{toc}{subsection}{
    \begin{CJK}{UTF8}{ipxm}~つもりです\end{CJK}: 
    Menunjukkan kemauan dan keinginan
}
Bentuk:
\begin{enumerate}
    \item Kata Kerja (辞書形) + つもりです
\end{enumerate}
Contoh: 
\begin{enumerate}
    \item あそこに座るつもりです。
    \\ (Aku) mau duduk di sana.
    \item 首が痛いから、あたらしいまくらを買うつもりです。
    \\ Karena leherku sakit, (aku) mau beli bantal baru.
\end{enumerate}

\vspace{0.2cm}\hrule height 1pt\vspace{0.2cm}

% =====================================================
\newpage
\subsection*{
    ~よてい \\
    Menjelaskan suatu rencana
}
\addcontentsline{toc}{subsection}{
    \begin{CJK}{UTF8}{ipxm}~よてい\end{CJK}: 
    Menjelaskan suatu rencana
}
Bentuk:
\begin{enumerate}
    \item Kata Kerja Bentuk Kasual (ふつう形) + よてい
\end{enumerate}
Contoh: 
\begin{enumerate}
    \item 今年は中国に帰るよていです。
    \\ Tahun ini saya berencana pulang ke China.
    \item わたしは家族と旅行にいくよていです。
    \\ Saya berencana liburan dengan keluarga.
    \item わたしは海に行くよていです。
    \\ Saya berencana pergi ke laut.
\end{enumerate}

\vspace{0.2cm}\hrule height 1pt\vspace{0.2cm}

% =====================================================

\subsection*{
    ~にします \\
    Mengutarakan pilihan dari beberapa opsi
}
\addcontentsline{toc}{subsection}{
    \begin{CJK}{UTF8}{ipxm}~にします\end{CJK}: 
    Mengutarakan pilihan dari beberapa opsi
}
Bentuk:
\begin{enumerate}
    \item Kata Kerja Bentuk Kasual (ふつう形) + にします
    \item Kata Benda + にします
\end{enumerate}
Contoh: 
\begin{enumerate}
    \item 洋服にします。
    \\ (Aku) mau duduk di sana.
    \item タ飯はすき焼きにします。
    \\ Karena leherku sakit, (aku) mau beli bantal baru.
\end{enumerate}

\vspace{0.2cm}\hrule height 1pt\vspace{0.2cm}

% =====================================================
\newpage
\subsection*{
    ~て/でも \\
    Menyampaikan pengandaian yang bertentangan
}
\addcontentsline{toc}{subsection}{
    \begin{CJK}{UTF8}{ipxm}~て/でも\end{CJK}: 
    Menyampaikan pengandaian yang bertentangan
}
Bentuk:
\begin{enumerate}
    \item Kata Kerja Bentuk Te (て形) + も
    \item Kata Kerja Bentuk Nai (な\sout{い}形) + くても
    \item Kata Sifat (\sout{i}) + くてみ
    \item Kata Sifat (\sout{な}) + で/じゃなくて + も
    \item Kata Benda + で/じゃなくて + も
\end{enumerate}
Contoh: 
\begin{enumerate}
    \item うるさくても優しいお兄さんです。
    \\ Meskipun cerewet, tapi (aku) kakak yang baik.
    \item 食べたくても、熱かったらすぐたべない。
    \\ Meskipun ingin sekali makan, tapi kalau panas tidak langsung kumakan.
\end{enumerate}

\vspace{0.2cm}\hrule height 1pt\vspace{0.2cm}
