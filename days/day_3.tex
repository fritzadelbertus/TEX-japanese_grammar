\begin{flushright}
    \section*{\Large{Day 3 三日目 \\
    Pola tentang Perpindahan dan Keberadaan}}
    \addcontentsline{toc}{section}{
        Day 3 \begin{CJK}{UTF8}{ipxm}三日目\end{CJK}: 
        Pola tentang Perpindahan dan Keberadaan
    }
\end{flushright}

\subsection*{
    ~へ+来ます・行きます・帰ります \\ 
    Menyatakan arah tujuan
}
\addcontentsline{toc}{subsection}{
    \begin{CJK}{UTF8}{ipxm}~へ+来ます・行きます・帰ります\end{CJK}: 
    Menyatakan arah tujuan
}
Pola kalimat yang menyatakan arah pergerakan.\\
Bentuk:
\begin{enumerate}
    \item Kata Benda (tempat) + へ + 行きます・来ます・帰ります
\end{enumerate}
Contoh: 
\begin{enumerate}
    \item 何時に学校へ行きますか?
    \\ Pada jam berapa pergi ke sekolah?
    \item わたしは8時半に学校へ来ます。
    \\ Saya datang ke sekolah pada jam 08:30.
    \item 何時にうちへ帰りますか?
    \\ Pada jam berapa pulang ke rumah?
\end{enumerate}

\vspace{0.2cm}\hrule height 1pt\vspace{0.2cm}

% =====================================================

\subsection*{
    ~へ~に行きます/来ます/帰ります \\
    Pergi untuk melakukan sesuatu
}
\addcontentsline{toc}{subsection}{
    \begin{CJK}{UTF8}{ipxm}~へ~に行きます/来ます/帰ります\end{CJK}: 
    Pergi untuk melakukan sesuatu
}
Pola kalimat ini digunakan untuk menunjukkan kegiatan yang dilakukan
di tempat yang dituju.\\
Bentuk:
\begin{enumerate}
    \item Kata benda (tempat tujuan) + へ + Kata Kerja (\sout{ます}) + に + 
    行きます・来ます・帰ります
\end{enumerate}
Contoh: 
\begin{enumerate}
    \item この文法の本お借りに来ます。
    \\ Hari ini saya meminjam buku pola kalimat yang ini.
    \item そうですか?じゃ、すぐ読みに行きます。
    \\ Benarkah? Kalau begitu saya akan segera pergi untuk membacanya.
    \item 私は日本へ国際関係を勉強しに来ました。
    \\ Saya datang ke Jepang untuk belajar Hubungan Internasional.
\end{enumerate}

\vspace{0.2cm}\hrule height 1pt\vspace{0.2cm}

% =====================================================
\newpage
\subsection*{
    ~で \\
    Menjelaskan cara/alat yang digunakan
}
\addcontentsline{toc}{subsection}{
    \begin{CJK}{UTF8}{ipxm}~で\end{CJK}: 
    Menjelaskan cara/alat yang digunakan
}
Partikel "de" berarti "dengan menggunakan" dapat dipakai untuk 
menyatakan alat yang digunakan.\\
Bentuk:
\begin{enumerate}
    \item Kata Benda (cara/alat/kendaraan) + で + Kata Kerja
\end{enumerate}
Contoh: 
\begin{enumerate}
    \item アメリカ人はナイフとフォークでたべます。
    \\ Orang Amerika makan menggunakan pisau dan garpu.
    \item わたしはナイフで野菜お切ります。
    \\ Sata memotong sayur menggunakan pisau.
    \item 先月、かぞくはひこうきでポンゴランドへ帰ります。
    \\ Bulan lalu keluarga (saya) pulang ke Pongoland dengan (naik) pesawat.
\end{enumerate}

\vspace{0.2cm}\hrule height 1pt\vspace{0.2cm}

% =====================================================

\subsection*{
    ~で \\
    Menjelaskan Kosakata dalam Bahasa Tertentu
}
\addcontentsline{toc}{subsection}{
    \begin{CJK}{UTF8}{ipxm}~で\end{CJK}: 
    Menjelaskan Kosakata dalam Bahasa Tertentu
}
Salah satu fungsi partikel "で" adalah untuk menjelaskan sesuatu dalam 
lingkup tertentu, contohnyaa "dalam bahasa...".\\
Bentuk:
\begin{enumerate}
    \item Bahasa ~+で
\end{enumerate}
Contoh: 
\begin{enumerate}
    \item 【ありがとうございます】はスペイン語で何ですか?
    \\ "Arigatou gozaimasu" dalam bahasa Spanyol apa?
    \item "Sorry" は日本語で【すみません】です。
    \\ "Sorry" dalam bahasa Jepang adalah "Sumimasen".
    \item 【お休みなさい】はインドネシア語で "Selamat tidur" です。
    \\ "Oyasuminasai" dalam bahasa Indonesia adalah "Selamat tidur".
\end{enumerate}

\vspace{0.2cm}\hrule height 1pt\vspace{0.2cm}

% =====================================================
\newpage
\subsection*{
    ~があります・います \\
    Menyatakan keberadaan \& kepemilikan
}
\addcontentsline{toc}{subsection}{
    \begin{CJK}{UTF8}{ipxm}~があります・います\end{CJK}: 
    Menyatakan keberadaan \& kepemilikan
}
Pola kalimat ini digunakan untuk menyatakan keberadaan. あります digunakan 
untuk benda mati dan makhluk hidup yang tidak bergerak sedangkan います 
digunakan untuk makhluk hidup.\\
Bentuk:
\begin{enumerate}
    \item Kata Benda (objek) + が + あります
    \item Kata Benda (objek) + が + います
\end{enumerate}
Contoh: 
\begin{enumerate}
    \item 手紙があります。
    \\ Ada surat.
    \item お金がありますか?
    \\ Apakah (kamu) punya uang?
    \item ねこがいます。
    \\ Ada (punya) kucing.
\end{enumerate}

\vspace{0.2cm}\hrule height 1pt\vspace{0.2cm}

% =====================================================

\subsection*{
    ~に~があります・います \\
    Keberadaan benda bergerak/tidak bergerak di suatu tempat
}
\addcontentsline{toc}{subsection}{
    \begin{CJK}{UTF8}{ipxm}~に~があります・います\end{CJK}: 
    Keberadaan benda bergerak/tidak bergerak di suatu tempat
}
Bentuk:
\begin{enumerate}
    \item Kata Benda (tempat) + に + Kata Benda (objek) + が + あります・います
\end{enumerate}
Contoh: 
\begin{enumerate}
    \item 今、きょうしつにわたしのかばんがありますか?
    \\ Sekarang apakah di kelas ada tas saya?
    \item マリオさん、かいぎしつに部長がいますか?
    \\ Pak Mario, apakah di ruang rapat ada Manager?
    \item お手洗いにだれもいません。
    \\ Di kamar mandi tidak ada siapapun.
\end{enumerate}

\vspace{0.2cm}\hrule height 1pt\vspace{0.2cm}

% =====================================================
\newpage
\subsection*{
    ~は~にあります・います \\
    Menyatakan tempat keberadaan topik bahasan
}
\addcontentsline{toc}{subsection}{
    \begin{CJK}{UTF8}{ipxm}~は~にあります・います\end{CJK}: 
    Menyatakan tempat keberadaan topik bahasan
}
Pola kalimat ini menyatakan tempat keberadaan topik bahasan
Bentuk:
\begin{enumerate}
    \item Kata Benda + は + Kata Benda (tempat) + に + あります・います
\end{enumerate}
Contoh: 
\begin{enumerate}
    \item しばちゃんはどこにいますか?ガレージの中にいますか?
    \\ Shiba ada dimana? Apakah ada di dalam garasi?
    \item しばちゃんの家はうちろにあります。
    \\ Rumah Shiba ada di belakang.
\end{enumerate}

\vspace{0.2cm}\hrule height 1pt\vspace{0.2cm}

% =====================================================

\subsection*{
    ~の~に~があります・います \\
    Menyatakan posisi benda atau makhluk hidup
}
\addcontentsline{toc}{subsection}{
    \begin{CJK}{UTF8}{ipxm}~の~に~があります・います\end{CJK}: 
    Menyatakan posisi benda atau makhluk hidup
}
Pola kalimat ini untuk menunjukkan posisi suatu benda secara spesifik
Bentuk:
\begin{enumerate}
    \item Kata Benda + の + Posisi + に + Kata Tanya + があります・います
    \item Kata Benda + は + Tempat + の + Posisi + に + あります・います
\end{enumerate}
Contoh: 
\begin{enumerate}
    \item あの机の上に何がありますか?
    \\ Di atas meja itu ada apa?
    \item 教室の中にだれがいますか?
    \\ Di dalam kelas ada siapa?
    \item ねこはつくえの上にいます。
    \\ Kucing ada di atas meja.
\end{enumerate}

\vspace{0.2cm}\hrule height 1pt\vspace{0.2cm}