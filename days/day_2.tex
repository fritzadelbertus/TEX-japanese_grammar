\begin{flushright}
    \section*{\Large{Day 2 二日目 \\
    Pola tentang Waktu dan Memberikan Pilihan}}
    \addcontentsline{toc}{section}{
        Day 2 \begin{CJK}{UTF8}{ipxm}二日目\end{CJK}: 
        Pola tentang Waktu dan Memberikan Pilihan
    }
\end{flushright}

\subsection*{
    ~時~分 \\ 
    Menyatakan jam dan menit
}
\addcontentsline{toc}{subsection}{
    \begin{CJK}{UTF8}{ipxm}~時~分\end{CJK}: 
    Menyatakan jam dan menit
}
Pola kalimat yang menyatakan jam dan menit.\\
Bentuk:
\begin{enumerate}
    \item Angka Jam + 時 + Angka Menit + 分
\end{enumerate}
Contoh: 
\begin{enumerate}
    \item 学校は午前9時15分からです。
    \\ Sekolah dari jam 9:15 pagi.
    \item 休みは午前11時30分です。
    \\ Istirahat jam 11:30 pagi.
    \item 晩ごはんは午後7時10分です。
    \\ Makan malam jam 7:10 malam.
\end{enumerate}

\vspace{0.2cm}\hrule height 1pt\vspace{0.2cm}

% =====================================================

\subsection*{
    ~から~まで \\
    Menyatakan rentang waktu
}
\addcontentsline{toc}{subsection}{
    \begin{CJK}{UTF8}{ipxm}~から~まで\end{CJK}: 
    Menyatakan rentang waktu
}
Pola kalimat ini digunakan untuk menunjukkan rentang waktu "dari.. sampai..".\\
Bentuk:
\begin{enumerate}
    \item Keterangan Waktu (dari) + から + Keterangan Waktu (sampai) + まで
\end{enumerate}
Contoh: 
\begin{enumerate}
    \item 学校は7時から2時までです。
    \\ Sekolah dari jam 7 sampai jam 2.
    \item 休みは9時から9時30分までです。
    \\ Libur dari jam 9 sampai jam 9:30.
    \item 休みは土曜日から日曜日までです。
    \\ Libur dari hari Sabtu sampai hari Minggu.
\end{enumerate}

\vspace{0.2cm}\hrule height 1pt\vspace{0.2cm}

% =====================================================
\newpage
\subsection*{
    ~に \\
    Menjelaskan waktu ketika melakukan suatu aktivitas
}
\addcontentsline{toc}{subsection}{
    \begin{CJK}{UTF8}{ipxm}~に\end{CJK}: 
    Menjelaskan waktu ketika melakukan suatu aktivitas
}
Partikel "ni" berarti "pada" untuk menunjukan waktu ketika 
melakukan aktivitas.\\
Bentuk:
\begin{enumerate}
    \item Keterangan waktu yang spesifik + に + Kata Kerja
\end{enumerate}
Contoh: 
\begin{enumerate}
    \item 午前6時に起きます。
    \\ (Saya) bangun jam 6 pagi.
    \item 午後4時に帰ります。
    \\ (Saya) pulang jam 4 siang.
    \item 午後8時に寝ます。
    \\ (Saya) tidur jam 8 malam.
\end{enumerate}

\vspace{0.2cm}\hrule height 1pt\vspace{0.2cm}

% =====================================================

\subsection*{
    ~を \\
    Menyatakan aktivitas atau pekerjaan yang memerlukan "objek"
}
\addcontentsline{toc}{subsection}{
    \begin{CJK}{UTF8}{ipxm}~を\end{CJK}: 
    Menyatakan aktivitas atau pekerjaan yang memerlukan "objek"
}
Fungsi partikel を adalah untuk menyatakan aktivitas yang memerlukan 
"objek".\\
Bentuk:
\begin{enumerate}
    \item Kata Benda + を + Kata Kerja
\end{enumerate}
Contoh: 
\begin{enumerate}
    \item いつも6時にシャワーをあびます。
    \\ Saya selalu mandi jam 6.
    \item 毎朝、朝ご飯を食べます・
    \\ Setiap pagi selalu sarapan.
    \item ぎゅうにゅうお飲みます。
    \\ Minum susu.
\end{enumerate}

\vspace{0.2cm}\hrule height 1pt\vspace{0.2cm}

% =====================================================
\newpage
\subsection*{
    ~か~か \\
    Memberikan pilihan
}
\addcontentsline{toc}{subsection}{
    \begin{CJK}{UTF8}{ipxm}~か~か\end{CJK}: 
    Memberikan pilihan
}
Pola kalimat ini digunakan untuk memberikan antara pilihan 1 
dan pilihan 2.\\
Bentuk:
\begin{enumerate}
    \item Kalimat 1 + か + Kalimat 2 + か
\end{enumerate}
Contoh: 
\begin{enumerate}
    \item ヒラさんは1年生ですか、2年生ですか?
    \\ Apakah Hira kelas 1 atau kelas 2?
    \item 最後の番号は8ですか、0ですか?
    \\ Nomor terakhir 8 atau 0?
\end{enumerate}

\vspace{0.2cm}\hrule height 1pt\vspace{0.2cm}

% =====================================================

\subsection*{
    ~ぐらいかかります \\
    Waktu yang diperlukan
}
\addcontentsline{toc}{subsection}{
    \begin{CJK}{UTF8}{ipxm}~ぐらいかかります\end{CJK}: 
    Waktu yang diperlukan
}
Pola kalimat ini digunakan untuk menunjukkan jumlah waktu 
yang diperlukan. Dalam bahasa Indonesia bisa diartikan 
"kira-kira".
Bentuk:
\begin{enumerate}
    \item Kata Bantu Bilangan + ぐらいかかります
\end{enumerate}
Contoh: 
\begin{enumerate}
    \item 家から学校まで1時間ぐらいかかります。
    \\ Dari rumah sampai sekolah kira-kira memerlukan 1 jam.
    \item 学校まで新幹線で30分ぐらいかかります。
    \\ Ke sekolah dengan Shinkansen kira-kira memerlukan waktu 30 menit.
    \item ヒラさんの家からわたしの家まで20分ぐらいかかります。
    \\ Dari rumah Hira ke rumah saya kira-kira memerlukan waktu 20 menit.
\end{enumerate}

\vspace{0.2cm}\hrule height 1pt\vspace{0.2cm}