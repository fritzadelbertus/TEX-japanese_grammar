\begin{flushright}
    \section*{\Large{Day 6 六日目 \\
    Pola ajakan dan bentuk て - Part 2}}
    \addcontentsline{toc}{section}{
        Day 6 \begin{CJK}{UTF8}{ipxm}六日目\end{CJK}: 
        Pola ajakan dan bentuk Te - Part 2
    }
\end{flushright}

\subsection*{
    ~てもいいです(か) \\ 
    Meminta atau memberi ijin
}
\addcontentsline{toc}{subsection}{
    \begin{CJK}{UTF8}{ipxm}~てもいいです(か)\end{CJK}: 
    Meminta atau memberi ijin
}
Bentuk:
\begin{enumerate}
    \item Kata Kerja (て形) + も + いい + です/ですか
\end{enumerate}
Contoh: 
\begin{enumerate}
    \item 一緒に作ってもいいですか?
    \\ Apa boleh membuat bersama?
    \item パンケーキに10にの卵を全部入れてもいいですか?
    \\ Apakah 10 butir telurnya boleh dimasukkan semua ke pancake?
    \item いいえ、メープルシロップおかけてください。でも、
    はちみつおかけてもいいですよ。
    \\ Tiduk, beri sirup maple. Tapi memberi madu juga boleh.
\end{enumerate}

\vspace{0.2cm}\hrule height 1pt\vspace{0.2cm}

% =====================================================

\subsection*{
    ~てはいけません \\
    Menyampaikan Larangan
}
\addcontentsline{toc}{subsection}{
    \begin{CJK}{UTF8}{ipxm}~てはいけません\end{CJK}: 
    Menyampaikan Larangan
}
Bentuk:
\begin{enumerate}
    \item Kata Kerja (て形) + は + いけません
\end{enumerate}
Contoh: 
\begin{enumerate}
    \item 赤いボタンを押してはいけません。
    \\ Tidak boleh menekan tombol merah.
    \item 赤いボタンのとなりも押してはいけません。
    \\ Sebelah tombol merah juga tidak boleh ditekan.
\end{enumerate}

\vspace{0.2cm}\hrule height 1pt\vspace{0.2cm}

% =====================================================
\newpage
\subsection*{
    ~て \\
    Menyampaikan 2 aktifitas atau lebih
}
\addcontentsline{toc}{subsection}{
    \begin{CJK}{UTF8}{ipxm}~て\end{CJK}: 
    Menyampaikan 2 aktifitas atau lebih
}
Aktifitas yang dijelaskan disini dilakukan secara berurut, 
(setalah melakukan ..., lalu melakukan..., lalu...).\\
Bentuk:
\begin{enumerate}
    \item Kata Kerja (て形) + Kata Kerja (て形) + ~
\end{enumerate}
Contoh: 
\begin{enumerate}
    \item やさいをあらって、きって、されにおいてください。
    パーティーへ来ませんか?
    \\ Tolong sayurnya dicuci, dipotong, (lalu) taruhlah di piring.
    \item しょうゆをいれて、ぶたにくをお入れてください。
    \\ Masukkan kecap asin (lalu) masukkan daging babi.
    \item 皿を取ってここに置いてください。
    \\ Tolong ambilkan piring lalu tolong letakkan di sini.
    \item ちょっとさらをここに置いて、ポンゴさんとリーさんを呼びますね。
    \\ Piringnya saya taruh sebentar disini, lalu saya panggil Pongo dan Lee.
\end{enumerate}

\vspace{0.2cm}\hrule height 1pt\vspace{0.2cm}

% =====================================================

\subsection*{
    ~くて/で \\
    Menyebutkan dua kata sifat
}
\addcontentsline{toc}{subsection}{
    \begin{CJK}{UTF8}{ipxm}~くて/で\end{CJK}: 
    Menyebutkan dua kata sifat
}
Ditambahkan selain pada sifat terkahir.\\
Bentuk:
\begin{enumerate}
    \item Kata Sifat (\sout{い}) + くて + Kata Sifat
    \item Kata Sifat (な) + で + Kata Sifat
\end{enumerate}
Contoh: 
\begin{enumerate}
    \item しずかさんはしずかで、おとなしいじょせいです。
    \\ Teman saya memberi saya jeruk. Silahkan dimakan.
    \item それに頭がよくて有名です。
    \\ Teman saya memberi saya jeruk. Silahkan dimakan.
\end{enumerate}

\vspace{0.2cm}\hrule height 1pt\vspace{0.2cm}

% =====================================================
\newpage
\subsection*{
    ~に~回 \\
    Menyatakan intensitas aktifitas pada suatu waktu
}
\addcontentsline{toc}{subsection}{
    \begin{CJK}{UTF8}{ipxm}~に~回\end{CJK}: 
    Menyatakan intensitas aktifitas pada suatu waktu
}
Bentuk:
\begin{enumerate}
    \item Kata Benda (rentang waktu) + に + Kata Benda (intensitas)
    + 回 + Kata Kerja
\end{enumerate}
Contoh: 
\begin{enumerate}
    \item 台所はきれいですね。いちにちに何回台所を掃除しますか?
    \\ Dapurnya bersih ya. Dalam satu hari membersihkan dapur berapa kali?
    \item いちにちに2回台所を掃除します。
    \\ Dalam 1 hari (saya) membersihkan dapur 2 kali.
    \item わたしはきれいなものが好きです。いちにちに2回お皿お洗います。
    \\ Saya suka barang yang bersih. Dalam satu hari saya mencuci piring 2 kali.
\end{enumerate}

\vspace{0.2cm}\hrule height 1pt\vspace{0.2cm}

% =====================================================

\subsection*{
    ~を/が \\
    Menyatakan satuan jumlah/waktu
}
\addcontentsline{toc}{subsection}{
    \begin{CJK}{UTF8}{ipxm}~を/が\end{CJK}: 
    Menyatakan satuan jumlah/waktu
}
Bentuk:
\begin{enumerate}
    \item Kata Benda + を/が + Keterangan Bilangan + Kata Kerja
\end{enumerate}
Contoh: 
\begin{enumerate}
    \item はい、友だちにコップを2つもらいました。
    \\ Iya, saya mendapat 2 buah gelas dari teman.
    \item ご家族が何人いますか?家族が4人います。
    \\ Keluargamu ada berapa orang? Ada 4 orang.
    \item ーヵ月に何回かぞくが来ますか?
    \\ Dalam 1 bulan keluargamu datang berapa kali?
    \item -ヵ月じゃありませによ。1年に1回ぐらい来ます。
    \\ Bukan dalam 1 bulan ya. Dalam 1 tahun datang kira-kira 1x.
\end{enumerate}

\vspace{0.2cm}\hrule height 1pt\vspace{0.2cm}
