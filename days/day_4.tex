\begin{flushright}
    \section*{\Large{Day 4 四日目 \\
    Pola tentang Keberadaan yang disertai Aktivitas}}
    \addcontentsline{toc}{section}{
        Day 4 \begin{CJK}{UTF8}{ipxm}四日目\end{CJK}: 
        Pola tentang Keberadaan yang disertai Aktivitas
    }
\end{flushright}

\subsection*{
    ~で/に~があります \\ 
    Menjelaskan keberadaan yang disertai aktivitas/tidak
}
\addcontentsline{toc}{subsection}{
    \begin{CJK}{UTF8}{ipxm}~で/に~があります\end{CJK}: 
    Menjelaskan keberadaan yang disertai aktivitas/tidak
}
Bentuk:
\begin{enumerate}
    \item Kata Benda (tempat) + で + Kata Benda (kegiatan) + が + あります
    \item Kata Benda (tempat) + に + Kata Benda + が + あります
\end{enumerate}
Contoh: 
\begin{enumerate}
    \item となりのうちでたんじょうびのパーティーがありますから、
    男の子がたくさんいます。
    \\ Karena di rumah sebelah ada pesta ulang tahun, sehingga ada banyak anak 
    laki-laki.
    \item こうえんでおちゃかいがありますから、わたしのこどももさんかします。
    \\ Karena di taman ada acara minum teh, anakku pun ikut serta.
    \item 私のうちに二つへやがあります。
    \\ Di rumah saya ada 2 kamar.
\end{enumerate}

\vspace{0.2cm}\hrule height 1pt\vspace{0.2cm}

% =====================================================

\subsection*{
    Mendeskripsikan sifat suatu benda
}
\addcontentsline{toc}{subsection}{
    Mendeskripsikan sifat suatu benda
}
Bentuk:
\begin{enumerate}
    \item Kata Benda + は + Kata Sifat i/na + です
\end{enumerate}
Contoh: 
\begin{enumerate}
    \item 今日はあついですね。
    \\ Hari ini panas ya.
    \item すいかはおいしかったです。
    \\ Semangka enak.
    \item 花火はきれいですね。ええ、見ましょう。
    \\ Kembang api indah ya. Ya, mari melihat.
    \item ゆかたはいいですね。
    \\ Yukata bagus ya.
    \item パンは便利ですから。
    \\ Karena roti praktis.
\end{enumerate}

\vspace{0.2cm}\hrule height 1pt\vspace{0.2cm}

% =====================================================
\newpage
\subsection*{
    Mendeskripsikan sifat menerangkan suatu benda
}
\addcontentsline{toc}{subsection}{
    Mendeskripsikan sifat menerangkan suatu benda
}
Bentuk:
\begin{enumerate}
    \item Kata Sifat i-い + Kata Benda
    \item Kata Sifat na-な + な + Kata Benda
\end{enumerate}
Contoh: 
\begin{enumerate}
    \item そうですね。じゃ、きれいなメイリンさんにあげます。
    \\ Benar ya. Kalau begitu saya berikan kepada Meilin yang cantik.
    \item いっしょに面白い(おもしろい)映画お見ませんか?
    \\ Maukah menonton film yang lucu bersama-sama?
    \item 好きなえいがはなんですか?
    \\ Film yang disukai apa?
\end{enumerate}

\vspace{0.2cm}\hrule height 1pt\vspace{0.2cm}

% =====================================================

\subsection*{
    ~の中で~が一番 \\
    Menyatakan Keadaan Suatu Benda dari Suatu Cakupan Tertentu
}
\addcontentsline{toc}{subsection}{
    \begin{CJK}{UTF8}{ipxm}~の中で~が一番\end{CJK}: 
    Menyatakan Keadaan Suatu Benda dari Suatu Cakupan Tertentu
}
Bentuk:
\begin{enumerate}
    \item Kata Benda (ruang lingkup) + の中で + Kata Benda 
    + が一番 + Kata Sifat
\end{enumerate}
Contoh: 
\begin{enumerate}
    \item  メイリンさん、かぞくの中でだれが一番きびしいですか?
    \\ Meilin, di antara semua anggota keluarga, siapa yang paling galak?
    \item かぞくで父が一番きびしいです。
    \\ Di dalam keluarga, ayah yang paling galak.
    \item ともだちの中でだれが一番わかいですか?
    \\ Di antara teman-teman semua, siapa yang paling muda?
    \item ともだちでしずかさんが一番わかいです。
    \\ Shizuka yang paling muda.
    \item のみものの中で何が一番好きですか?
    \\ Di antara minuman, apa yang paling suka?
\end{enumerate}

\vspace{0.2cm}\hrule height 1pt\vspace{0.2cm}

