\begin{flushright}
    \section*{\Large{Day 10 十日目 \\
    Pola Bentuk Lampau dan Kata Kerja Bentuk Kasual}}
    \addcontentsline{toc}{section}{
        Day 10 \begin{CJK}{UTF8}{ipxm}十日目\end{CJK}: 
        Pola Bentuk Lampau dan Kata Kerja Bentuk Kasual
    }
\end{flushright}

\subsection*{
    ~たり \\ 
    Menjelaskan beberapa kegiatan dalam kalimat
}
\addcontentsline{toc}{subsection}{
    \begin{CJK}{UTF8}{ipxm}~たり\end{CJK}: 
    Menjelaskan beberapa kegiatan dalam kalimat
}
Semua kata kerja ditambahkan bentuk (たり).
Bentuk:
\begin{enumerate}
    \item Kata Kerja Bentuk Lampau (た形)り + 
    Kata Kerja Bentuk Lampau (た形) + します
\end{enumerate}
Contoh: 
\begin{enumerate}
    \item 休みの日はユーチューブをみたり、散歩をしたりします。ポンゴさんは?
    \\ Hari libur saya menonton youtube, berjalan jalan. Kalau Pongo?
    \item フイログをみたり、部屋の掃除をしたりします。
    \\ Nonton vlog, membersihkan kamar.
    \item 夏休みはこの海の写真を撮ったり、海で遊んだりします。
    \\ Liburan musim panas, saya mengambil foto laut ini, bermain di laut.
\end{enumerate}

\vspace{0.2cm}\hrule height 1pt\vspace{0.2cm}

% =====================================================

\subsection*{
    ~なる \\
    Mendeskripsikan perubahan keadaan
}
\addcontentsline{toc}{subsection}{
    \begin{CJK}{UTF8}{ipxm}~なる\end{CJK}: 
    Mendeskripsikan perubahan keadaan
}
Bentuk:
\begin{enumerate}
    \item Kata Sifat (\sout{い}) + くなります
    \item Kata Sifat (\sout{な}) + になります
\end{enumerate}
Contoh: 
\begin{enumerate}
    \item 来週、試験がありますから、べんきょうでいそがしくなります。
    \\ Minggu depang ada ujian, jadi mulai sibuk dengan belajar.
    \item 部屋でも大丈夫です。一度スポーツをしないと体が弱くなります。
    \\ Di kamar pun tidak masalah. Sekali tidak berolahraga tubuh menjadi lemah.
    \item コーヒーの缶ちスナック袋をまだ捨てないから、汚くなります。
    \\ Kaleng kopi dan kantong snack masih belum dibuang, jadinya kotor.
\end{enumerate}

\vspace{0.2cm}\hrule height 1pt\vspace{0.2cm}

% =====================================================
\newpage
\subsection*{
    ~と思います \\
    Menyampaikan hal yang ada dalam pikiran
}
\addcontentsline{toc}{subsection}{
    \begin{CJK}{UTF8}{ipxm}~と思います\end{CJK}: 
    Menyampaikan hal yang ada dalam pikiran
}
Bentuk:
\begin{enumerate}
    \item Kata Kerja + と思います
    \item Kata Sifat (\sout{な}) + と思います
    \item Kata Sifat (\sout{な}) + だ + と思います
    \item Kata Benda + だ + と思います
\end{enumerate}
Contoh: 
\begin{enumerate}
    \item 今年の3月は去年より暖かいから、桜が早く咲くと思います。
    \\ Karena kontesnya Rabu pekan depan, harus hafal paling lambat hari minggu.
    \item 昼はいいと思います。ゆっくり花見ができますね。
    \\ Sepertinya bagus siang. Bisa melihat hanami dengan tenang.
\end{enumerate}

\vspace{0.2cm}\hrule height 1pt\vspace{0.2cm}

% =====================================================

\subsection*{
    ~てから \\
    Menerangkan tindakan yang dilakukan setelah melakukan sesuatu
}
\addcontentsline{toc}{subsection}{
    \begin{CJK}{UTF8}{ipxm}~てから\end{CJK}: 
    Menerangkan tindakan yang dilakukan setelah melakukan sesuatu
}
Bentuk:
\begin{enumerate}
    \item Kata Kerja 1 (て形) + から + Kata Kerja 2
\end{enumerate}
Contoh: 
\begin{enumerate}
    \item リーさんは手を洗ってから食べますか?
    \\ Lee tangannya dicuci dulu baru makan ya?
    \item もちろん!汚いから手を洗ってから食べるよ!
    \\ Iyalah! Karena kotor tangan dicuci baru makan ya!
    \item いいえ、一時間ぐらい待ってから歯を磨きます。
    \\ Tidak, setelah menunggu sekitar 1 jam saya menyikat gigi.
    \item 私はズボンをぬいでからあびます。
    \\ Saya lepas celana dulu setelah itu mandi.
\end{enumerate}

\vspace{0.2cm}\hrule height 1pt\vspace{0.2cm}

% =====================================================
\newpage
\subsection*{
    ~たことがあります \\
    Menjelaskan hal yang pernah dilakukan/dialami
}
\addcontentsline{toc}{subsection}{
    \begin{CJK}{UTF8}{ipxm}~たことがあります\end{CJK}: 
    Menjelaskan hal yang pernah dilakukan/dialami
}
Bentuk:
\begin{enumerate}
    \item Kata Kerja Bentuk Lampau (た形) + こと + が + あります
\end{enumerate}
Contoh: 
\begin{enumerate}
    \item 日本のお酒を飲んだんことがあります。
    \\ Apakah kamu pernah minum sake jepang?
    \item お茶を飲んだこともありますか?
    \\ Apakah pernah minum teh hijau juga?
    \item お寿司も食べたことがあります。イカとサーモンが好きです。
    \\ Iya, aku juga pernah maka sushi. Suka salmon dan cumi.
\end{enumerate}

\vspace{0.2cm}\hrule height 1pt\vspace{0.2cm}
