\begin{flushright}
    \section*{\Large{Day 10 十日目 \\
    Pola Bentuk Lampau dan Kata Kerja Bentuk Kasual}}
    \addcontentsline{toc}{section}{
        Day 10 \begin{CJK}{UTF8}{ipxm}十日目\end{CJK}: 
        Pola Bentuk Lampau dan Kata Kerja Bentuk Kasual
    }
\end{flushright}

\subsection*{
    ~たり \\ 
    Menjelaskan beberapa kegiatan dalam kalimat
}
\addcontentsline{toc}{subsection}{
    \begin{CJK}{UTF8}{ipxm}~たり\end{CJK}: 
    Menjelaskan beberapa kegiatan dalam kalimat
}
Semua kata kerja ditambahkan bentuk (たり).
Bentuk:
\begin{enumerate}
    \item Kata Kerja Bentuk Lampau (た形)り + 
    Kata Kerja Bentuk Lampau (た形) + します
\end{enumerate}
Contoh: 
\begin{enumerate}
    \item 休みの日はユーチューブをみたり、散歩をしたりします。ポンゴさんは?
    \\ Hari libur saya menonton youtube, berjalan jalan. Kalau Pongo?
    \item フイログをみたり、部屋の掃除をしたりします。
    \\ Nonton vlog, membersihkan kamar.
    \item 夏休みはこの海の写真を撮ったり、海で遊んだりします。
    \\ Liburan musim panas, saya mengambil foto laut ini, bermain di laut.
\end{enumerate}

\vspace{0.2cm}\hrule height 1pt\vspace{0.2cm}

% =====================================================

\subsection*{
    ~なる \\
    Mendeskripsikan perubahan keadaan
}
\addcontentsline{toc}{subsection}{
    \begin{CJK}{UTF8}{ipxm}~なる\end{CJK}: 
    Mendeskripsikan perubahan keadaan
}
Bentuk:
\begin{enumerate}
    \item Kata Sifat (\sout{い}) + くなります
    \item Kata Sifat (\sout{な}) + になります
\end{enumerate}
Contoh: 
\begin{enumerate}
    \item 来週、試験がありますから、べんきょうでいそがしくなります。
    \\ Minggu depang ada ujian, jadi mulai sibuk dengan belajar.
    \item 部屋でも大丈夫です。一度スポーツをしないと体が弱くなります。
    \\ Di kamar pun tidak masalah. Sekali tidak berolahraga tubuh menjadi lemah.
    \item コーヒーの缶ちスナック袋をまだ捨てないから、汚くなります。
    \\ Kaleng kopi dan kantong snack masih belum dibuang, jadinya kotor.
\end{enumerate}

\vspace{0.2cm}\hrule height 1pt\vspace{0.2cm}

% =====================================================
\newpage
\subsection*{
    ~と思います \\
    Menyampaikan hal yang ada dalam pikiran
}
\addcontentsline{toc}{subsection}{
    \begin{CJK}{UTF8}{ipxm}~と思います\end{CJK}: 
    Menyampaikan hal yang ada dalam pikiran
}
Bentuk:
\begin{enumerate}
    \item Kata Kerja + と思います
    \item Kata Sifat (\sout{な}) + と思います
    \item Kata Sifat (\sout{な}) + だ + と思います
    \item Kata Benda + だ + と思います
\end{enumerate}
Contoh: 
\begin{enumerate}
    \item 今年の3月は去年より暖かいから、桜が早く咲くと思います。
    \\ Karena kontesnya Rabu pekan depan, harus hafal paling lambat hari minggu.
    \item 昼はいいと思います。ゆっくり花見ができますね。
    \\ Sepertinya bagus siang. Bisa melihat hanami dengan tenang.
\end{enumerate}

\vspace{0.2cm}\hrule height 1pt\vspace{0.2cm}

% =====================================================

\subsection*{
    ~と言います \\
    Menyebutkan sesuatu
}
\addcontentsline{toc}{subsection}{
    \begin{CJK}{UTF8}{ipxm}~と言います\end{CJK}: 
    Menyebutkan sesuatu
}
Bentuk:
\begin{enumerate}
    \item Kata Kerja Bentuk Kamus (辞書形) + と + 言います
\end{enumerate}
Contoh: 
\begin{enumerate}
    \item 日本人はお祈りをして、【いただきます】と言います。
    \\ Orang Jepang berdoa dan mengucapkan "itadakimasu" sebelum makan.
    \item 日本人は食べてから【ごちそうさまでした】と言います。
    \\ Orang Jepang setelah makan mengucapkan "gochisousamadeshita".
    \item いつも【ごちそうさまでした】と言いました。
    \\ Selalu mengucapkan "gochisousamadeshita".
    \item 母は食べてから片付けてくださいと言いました。
    \\ Ibu (saya) berkata bereskan (meja) setelah makan.
\end{enumerate}

\vspace{0.2cm}\hrule height 1pt\vspace{0.2cm}

% =====================================================
\newpage
\subsection*{
    ~でしょう \\
    Memastikan isi pembicaraan pada pembicara
}
\addcontentsline{toc}{subsection}{
    \begin{CJK}{UTF8}{ipxm}~でしょう\end{CJK}: 
    Memastikan isi pembicaraan pada pembicara
}
Bentuk:
\begin{enumerate}
    \item Kata Kerja Bentuk Kamus (辞書形) + でしょう
    \item Kata Sifat (い) + でしょう
    \item Kata Sifat (\sout{な}) + でしょう
    \item Kata Benda + でしょう 
\end{enumerate}
Contoh: 
\begin{enumerate}
    \item これは中国の地図でしょう。
    \\ Ini adalah peta Cina bukan?
    \item 中国は山も多いでしょう。
    \\ Cina juga banyak gunung bukan?
\end{enumerate}

\vspace{0.2cm}\hrule height 1pt\vspace{0.2cm}
