\begin{flushright}
    \section*{\Large{Day 5 五日目 \\
    Pola ajakan dan bentuk て - Part 1}}
    \addcontentsline{toc}{section}{
        Day 5 \begin{CJK}{UTF8}{ipxm}五日目\end{CJK}: 
        Pola ajakan dan bentuk Te - Part 1
    }
\end{flushright}

\subsection*{
    ~ましょう \\ 
    Mengajak
}
\addcontentsline{toc}{subsection}{
    \begin{CJK}{UTF8}{ipxm}~ましょう\end{CJK}: 
    Mengajak
}
Kalimat ajakan secara aktif atau menanggapi ajakan/usulan 
secara aktif.\\
Bentuk:
\begin{enumerate}
    \item Kata Kerja (\sout{ます}) + ましょう
\end{enumerate}
Contoh: 
\begin{enumerate}
    \item ええ、げんきんで払いましょう。
    \\ Ya, ayo membayar dengan uang tunai.
    \item むりょうですから、とりましょう。
    \\ Karena gratis, ayo kita mengambil.
\end{enumerate}

\vspace{0.2cm}\hrule height 1pt\vspace{0.2cm}

% =====================================================

\subsection*{
    ~ましょうか \\
    Menawarkan untuk melakukan sesuatu
}
\addcontentsline{toc}{subsection}{
    \begin{CJK}{UTF8}{ipxm}~ましょうか\end{CJK}: 
    Menawarkan untuk melakukan sesuatu
}
Digunakan untuk menawarkan diri atau ajakan untuk melakukan 
suatu aktivitas.\\
Bentuk:
\begin{enumerate}
    \item Kata Kerja (\sout{ます}) + ましょうか
\end{enumerate}
Contoh: 
\begin{enumerate}
    \item 重いから、私が持ちましょうか?
    \\ Karena berat boleh saya bawakan?
\end{enumerate}

\vspace{0.2cm}\hrule height 1pt\vspace{0.2cm}

% =====================================================
\newpage
\subsection*{
    ~ませんか \\
    Memberikan tawaran
}
\addcontentsline{toc}{subsection}{
    \begin{CJK}{UTF8}{ipxm}~ませんか\end{CJK}: 
    Memberikan tawaran
}
Pola kalimat ajakan/tawaran ini digunakan untuk mengajak 
lawan bicara yang belum tahu dia akan setuju atau tidak.\\
Bentuk:
\begin{enumerate}
    \item Kata Kerja (\sout{ます}) + ませんか
\end{enumerate}
Contoh: 
\begin{enumerate}
    \item わたしのたんじょうびはらいしゅうのすいようびです。
    パーティーへ来ませんか?
    \\ Ulang tahun saya hari Rabu minggu depan. 
    Maukah datang ke pesta?
    \item しずかちゃん、いっしょにパーティーへ行きませんか?
    \\ Shizuka, maukah pergi ke pesta bersama?
    \item でも、プレゼントを買いますから。一緒に買いませんか?
    \\ Tetapi, karena aku akan membeli hadiah, maukah membeli bersama?
\end{enumerate}

\vspace{0.2cm}\hrule height 1pt\vspace{0.2cm}

% =====================================================

\subsection*{
    ~くれます \\
    Memberikan sesuatu pada pembicara
}
\addcontentsline{toc}{subsection}{
    \begin{CJK}{UTF8}{ipxm}~くれます\end{CJK}: 
    Memberikan sesuatu pada pembicara
}
Bentuk:
\begin{enumerate}
    \item Subjek + は + Pembicara + に + Kata Benda + を + くれます
\end{enumerate}
Contoh: 
\begin{enumerate}
    \item 友達はわたしにみかんをくれました。どうぞ食べてください。
    \\ Teman saya memberi saya jeruk. Silahkan dimakan.
\end{enumerate}

\vspace{0.2cm}\hrule height 1pt\vspace{0.2cm}

% =====================================================

\subsection*{
    ~あげます \\
    Memberikan sesuatu pada orang lain
}
\addcontentsline{toc}{subsection}{
    \begin{CJK}{UTF8}{ipxm}~あげます\end{CJK}: 
    Memberikan sesuatu pada orang lain
}
Bentuk:
\begin{enumerate}
    \item Subjek + は + Penerima + に + Kata Benda + を + くれます
\end{enumerate}
Contoh: 
\begin{enumerate}
    \item 私はアグスさんに日本ごの辞書をあげます。
    \\ Saya memberikan kamus Bahasa Jepang kepada Agus.
\end{enumerate}

\vspace{0.2cm}\hrule height 1pt\vspace{0.2cm}

% =====================================================
\newpage
\subsection*{
    ~もらいます \\
    Menerima sesuatu dari orang lain
}
\addcontentsline{toc}{subsection}{
    \begin{CJK}{UTF8}{ipxm}~もらいます\end{CJK}: 
    Menerima sesuatu dari orang lain
}
Bentuk:
\begin{enumerate}
    \item Penerima + は + Pemberi + に + Kata Benda + を + くれます
\end{enumerate}
Contoh: 
\begin{enumerate}
    \item 私はデデさんに携帯電話をもらいます。
    \\ Saya menerima handphone dari Dede.
    \item 母の誕生日に母は父に花をもらいました。
    \\ Pada ulang tahun ibu, ibu menerima bunga dari ayah.
\end{enumerate}

\vspace{0.2cm}\hrule height 1pt\vspace{0.2cm}

% =====================================================

\subsection*{
    ~ています \\
    Menyatakan kondisi/aktifitas yang sedang berlangsung
}
\addcontentsline{toc}{subsection}{
    \begin{CJK}{UTF8}{ipxm}~ています\end{CJK}: 
    Menyatakan kondisi/aktifitas yang sedang berlangsung
}
Bentuk:
\begin{enumerate}
    \item Kata Kerja (て形) + います
\end{enumerate}
Contoh: 
\begin{enumerate}
    \item 雪が降りますね、かさおもっていますか?
    \\ Salju turun ya, apa kamu membawa payung?
    \item はい、かさおもっています。
    \\ Iya, saya membaca payung.
    \item 今日は寒いですから、コートおきています。
    \\ Karena hari ini dingin, saya memakai mantel.
    \item メイリンさん、ハイマートがしっていましか?
    \\ Meilin, apakah kamu tau Hi-Mart?
    \item みやもと先生はけっこんしていますから、おくさんとすんでいます。
    \\ Karena Guru Miyamoto sudah menikah, jadi tinggal dengan istrinya.
\end{enumerate}

\vspace{0.2cm}\hrule height 1pt\vspace{0.2cm}

% =====================================================
\newpage
\subsection*{
    ~てください \\
    Kalimat perintah atau meminta
}
\addcontentsline{toc}{subsection}{
    \begin{CJK}{UTF8}{ipxm}~てください\end{CJK}: 
    Kalimat perintah atau meminta
}
Bentuk:
\begin{enumerate}
    \item Kata Kerja (て形) + ください
\end{enumerate}
Contoh: 
\begin{enumerate}
    \item もう8時ですよ。早く晩ご飯食べてください。
    \\ Sudah jam 8 loh. Segeralah makan malam.
    \item じゃ、刺身を買ってください。
    \\ Kalau begitu, belilah sashimi.
\end{enumerate}

\vspace{0.2cm}\hrule height 1pt\vspace{0.2cm}

% =====================================================

\subsection*{
    ~ておきます \\
    Mempersiapkan
}
\addcontentsline{toc}{subsection}{
    \begin{CJK}{UTF8}{ipxm}~ておきます\end{CJK}: 
    Mempersiapkan
}
Bentuk:
\begin{enumerate}
    \item Kata Kerja (て形) + おきます
\end{enumerate}
Contoh: 
\begin{enumerate}
    \item パーティーは9時にしますから、いまから食べ物を料理しておきます。
    \\ Karena pestanya akan dilakukan jam 9, dari sekarang mempersiapkan makanan.
    \item そろそろばんごはんですよ、お皿おならべておいてください。
    \\ Sebentar lagi makan malam, tolong mulai susun piringnya.
\end{enumerate}

\vspace{0.2cm}\hrule height 1pt\vspace{0.2cm}